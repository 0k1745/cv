\documentclass[10pt, letterpaper]{article}

% Packages:
\usepackage[
    ignoreheadfoot, % set margins without considering header and footer
    top=2 cm, % seperation between body and page edge from the top
    bottom=2 cm, % seperation between body and page edge from the bottom
    left=2 cm, % seperation between body and page edge from the left
    right=2 cm, % seperation between body and page edge from the right
    footskip=1.0 cm, % seperation between body and footer
    % showframe % for debugging 
]{geometry} % for adjusting page geometry
\usepackage[explicit]{titlesec} % for customizing section titles
\usepackage{tabularx} % for making tables with fixed width columns
\usepackage{array} % tabularx requires this
\usepackage[dvipsnames]{xcolor} % for coloring text
\definecolor{primaryColor}{RGB}{0, 79, 144} % define primary color
\usepackage{enumitem} % for customizing lists
\usepackage{fontawesome5} % for using icons
\usepackage{amsmath} % for math
\usepackage[
    pdftitle={Stéphane Saunier's CV},
    pdfauthor={Stéphane Saunier},
    pdfcreator={LaTeX with RenderCV},
    colorlinks=true,
    urlcolor=primaryColor
]{hyperref} % for links, metadata and bookmarks
\usepackage[pscoord]{eso-pic} % for floating text on the page
\usepackage{calc} % for calculating lengths
\usepackage{bookmark} % for bookmarks
\usepackage{lastpage} % for getting the total number of pages
\usepackage{changepage} % for one column entries (adjustwidth environment)
\usepackage{paracol} % for two and three column entries
\usepackage{ifthen} % for conditional statements
\usepackage{needspace} % for avoiding page brake right after the section title
\usepackage{iftex} % check if engine is pdflatex, xetex or luatex

% Ensure that generate pdf is machine readable/ATS parsable:
\ifPDFTeX
    \input{glyphtounicode}
    \pdfgentounicode=1
    \usepackage[T1]{fontenc}
    \usepackage[utf8]{inputenc}
    \usepackage{lmodern}
\fi

\usepackage[default, type1]{sourcesanspro} 

% Some settings:
\AtBeginEnvironment{adjustwidth}{\partopsep0pt} % remove space before adjustwidth environment
\pagestyle{empty} % no header or footer
\setcounter{secnumdepth}{0} % no section numbering
\setlength{\parindent}{0pt} % no indentation
\setlength{\topskip}{0pt} % no top skip
\setlength{\columnsep}{0.15cm} % set column seperation
\makeatletter
\let\ps@customFooterStyle\ps@plain % Copy the plain style to customFooterStyle
\patchcmd{\ps@customFooterStyle}{\thepage}{
    \color{gray}\textit{\small Stéphane Saunier - Page \thepage{} of \pageref*{LastPage}}
}{}{} % replace number by desired string
\makeatother
\pagestyle{customFooterStyle}

\titleformat{\section}{
    % avoid page braking right after the section title
    \needspace{4\baselineskip}
    % make the font size of the section title large and color it with the primary color
    \Large\color{primaryColor}
}{
}{
}{
    % print bold title, give 0.15 cm space and draw a line of 0.8 pt thickness
    % from the end of the title to the end of the body
    \textbf{#1}\hspace{0.15cm}\titlerule[0.8pt]\hspace{-0.1cm}
}[] % section title formatting

\titlespacing{\section}{
    % left space:
    -1pt
}{
    % top space:
    0.3 cm
}{
    % bottom space:
    0.2 cm
} % section title spacing

% \renewcommand\labelitemi{$\vcenter{\hbox{\small$\bullet$}}$} % custom bullet points
\newenvironment{highlights}{
    \begin{itemize}[
        topsep=0.10 cm,
        parsep=0.10 cm,
        partopsep=0pt,
        itemsep=0pt,
        leftmargin=0.4 cm + 10pt
    ]
}{
    \end{itemize}
} % new environment for highlights

\newenvironment{highlightsforbulletentries}{
    \begin{itemize}[
        topsep=0.10 cm,
        parsep=0.10 cm,
        partopsep=0pt,
        itemsep=0pt,
        leftmargin=10pt
    ]
}{
    \end{itemize}
} % new environment for highlights for bullet entries


\newenvironment{onecolentry}{
    \begin{adjustwidth}{
        0.2 cm + 0.00001 cm
    }{
        0.2 cm + 0.00001 cm
    }
}{
    \end{adjustwidth}
} % new environment for one column entries

\newenvironment{twocolentry}[2][]{
    \onecolentry
    \def\secondColumn{#2}
    \setcolumnwidth{\fill, 4.5 cm}
    \begin{paracol}{2}
}{
    \switchcolumn \raggedleft \secondColumn
    \end{paracol}
    \endonecolentry
} % new environment for two column entries

\newenvironment{threecolentry}[3][]{
    \onecolentry
    \def\thirdColumn{#3}
    \setcolumnwidth{1 cm, \fill, 4.5 cm}
    \begin{paracol}{3}
    {\raggedright #2} \switchcolumn
}{
    \switchcolumn \raggedleft \thirdColumn
    \end{paracol}
    \endonecolentry
} % new environment for three column entries

\newenvironment{header}{
    \setlength{\topsep}{0pt}\par\kern\topsep\centering\color{primaryColor}\linespread{1.5}
}{
    \par\kern\topsep
} % new environment for the header

\newcommand{\placelastupdatedtext}{% \placetextbox{<horizontal pos>}{<vertical pos>}{<stuff>}
  \AddToShipoutPictureFG*{% Add <stuff> to current page foreground
    \put(
        \LenToUnit{\paperwidth-2 cm-0.2 cm+0.05cm},
        \LenToUnit{\paperheight-1.0 cm}
    ){\vtop{{\null}\makebox[0pt][c]{
        \small\color{gray}\textit{Last updated in November 2024}\hspace{\widthof{Last updated in November 2024}}
    }}}%
  }%
}%

% save the original href command in a new command:
\let\hrefWithoutArrow\href

% new command for external links:
\renewcommand{\href}[2]{\hrefWithoutArrow{#1}{\ifthenelse{\equal{#2}{}}{ }{#2 }\raisebox{.15ex}{\footnotesize \faExternalLink*}}}


\begin{document}
    \newcommand{\AND}{\unskip
        \cleaders\copy\ANDbox\hskip\wd\ANDbox
        \ignorespaces
    }
    \newsavebox\ANDbox
    \sbox\ANDbox{}

    \placelastupdatedtext
    \begin{header}
        \fontsize{30 pt}{30 pt}
        \textbf{Stéphane Saunier}

        \vspace{0.3 cm}

        \normalsize
        \mbox{{\footnotesize\faMapMarker*}\hspace*{0.13cm}Nantes, France}%
        \kern 0.25 cm%
        \AND%
        \kern 0.25 cm%
        \mbox{\hrefWithoutArrow{mailto:saunier.steph@gmail.com}{{\footnotesize\faEnvelope[regular]}\hspace*{0.13cm}saunier.steph@gmail.com}}%
        \kern 0.25 cm%
        \AND%
        \kern 0.25 cm%
        \mbox{\hrefWithoutArrow{tel:+33-6-86-12-80-49}{{\footnotesize\faPhone*}\hspace*{0.13cm}06 86 12 80 49}}%
        \kern 0.25 cm%
        \AND%
        \kern 0.25 cm%
        \mbox{\hrefWithoutArrow{https://linkedin.com/in/stephane-saunier}{{\footnotesize\faLinkedinIn}\hspace*{0.13cm}stephane-saunier}}%
        \kern 0.25 cm%
        \AND%
        \kern 0.25 cm%
        \mbox{\hrefWithoutArrow{https://github.com/Stephane-Saunier}{{\footnotesize\faGithub}\hspace*{0.13cm}Stephane-Saunier}}%
        \kern 0.25 cm%
        \AND%
        \kern 0.25 cm%
        \mbox{\hrefWithoutArrow{https://github.com/0k1745}{{\footnotesize\faGithub}\hspace*{0.13cm}0k1745}}%
    \end{header}

    \vspace{0.3 cm - 0.3 cm}


    \section{Experience}



        
        \begin{twocolentry}{
            Nantes, France

        May 2022 – present

        2 years 7 months
        }
            \textbf{Decathlon}, Senior Software Engineer
            \begin{highlights}
                \item At Decathlon, I have been able to implement and actively share what I’ve learned from past experiences. As a senior developer, my main role is mentoring junior developers and challenging proposed solutions from architects and the team. In a central role on the project, I coordinate and plan for the team to achieve high-quality software. My biggest success so far at Decathlon has been guiding our junior production expert, who successfully transitioned to a backend development role
                \item Leading backend development (Java, Spring Boot) on Kubernetes in Google Cloud
                \item Managing software organization, development tracking, and collaboration with stakeholders
                \item Collaborating closely with Engineering Manager (EM) with autonomy on backend decisions
                \item Supporting architectural decisions and providing technical guidance to the team
                \item Conducting bug analysis, testing, and quality assurance
                \item Authoring technical specifications and overseeing Level 3/4 support
            \end{highlights}
        \end{twocolentry}


        \vspace{0.2 cm}

        \begin{twocolentry}{
            Full remote, Tours, France

        Jan 2021 – May 2022

        1 year 4 months
        }
            \textbf{Île de France Mobilités, Worldline}, Lead Developer
            \begin{highlights}
                \item This project focuses on digitizing transit tickets for the Île-de-France region. We implemented a service orchestration system using Apache Camel to serve as the gateway for various data consumers. My team of four and I notably integrated Apple Pay, allowing us to coordinate external ticketing solutions, an internal authentication system (based on Keycloak), and systems for payment and digital ticket distribution
                \item Supported architectural studies and design decisions
                \item Managed development tracking and coordination
                \item Provided team guidance and support
                \item Developed features using Java, Spring, Hibernate, Maven
                \item Configured operational tools (Docker, Keycloak)
                \item Conducted testing and ensured quality assurance
                \item Scoped and authored specifications
                \item Delivered Level 4 support and maintenance
            \end{highlights}
        \end{twocolentry}


        \vspace{0.2 cm}

        \begin{twocolentry}{
            Hybrid, Tours, France

        May 2018 – Jan 2021

        2 years 8 months
        }
            \textbf{Trusted Authentication, Worldline}, Lead Developer
            \begin{highlights}
                \item Within the Trusted Authentication team, I held the position of lead developer/manager for the backend team. This R\&D project aimed to provide strong authentication for banks through mobile SDKs and backend services. When I joined, the team mostly comprised developers with limited experience. Working with my manager, we expanded the team with seasoned professionals to achieve the quality and reliability we needed. After three years of growth and development on this project, I decided to pursue new challenges
                \item Manage a team of 8
                \item Analyzed and designed core functionalities
                \item Coordinated development and implementation
                \item Assisted team with troubleshooting and support
                \item Conducted bug analysis and resolution
                \item Developed and tested features (Java, Spring, Hibernate, Maven)
                \item Scoped and authored specifications for product features
                \item Delivered Level 4 support and maintenance
            \end{highlights}
        \end{twocolentry}


        \vspace{0.2 cm}

        \begin{twocolentry}{
            Hybrid, Tours, France

        Jan 2013 – May 2018

        5 years 4 months
        }
            \textbf{Machine To Machine ,Worldline}, Software Engineer
            \begin{highlights}
                \item The M2M project is a multi-service initiative that manages a fleet of connected devices. As my first major Java enterprise project, I learned significantly from my peers, especially the senior developer who guided us. I proved myself in this team, eventually stepping into a Lead Developer role. Through my experience as a Scrum Master, I implemented agile methodologies, embracing the “you build it, you run it” culture that deeply shaped my approach to development
                \item Served as Scrum Master, facilitating agile development
                \item Developed backend components (Java, Spring, Hibernate, Maven)
                \item Conducted bug analysis and resolution
                \item Managed deployment and testing processes
                \item Provided Level 3 support and maintenance
            \end{highlights}
        \end{twocolentry}



    
    \section{Education}



        
        \begin{threecolentry}{\textbf{Eng.D.}}{
            Sept 2012
        }
            \textbf{Polytech Tours}, Computer Science
        \end{threecolentry}


    
    \section{Technologies}



        
        \begin{onecolentry}
            \textbf{Languages:} Java, SpringBoot, SQL
        \end{onecolentry}

        \vspace{0.2 cm}

        \begin{onecolentry}
            \textbf{Technologies:} GKE, PostgreSQL, MySQL, docker, git, Github, GitLab
        \end{onecolentry}


    

\end{document}